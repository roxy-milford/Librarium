\chapter{Комплексні числа і числа комплексної змінної}

\section{Комплексні числа. основні визначення}

\textbf{Уявна одиниця} --- це спеціальне число в математиці з незвичайними властивостями.

Властивість уявної одиниці $i^2 = -1$.

Якщо ми маємо щось схоже на $\dfrac{x}{iy}$, то, $\dfrac{x}{iy} = - \dfrac{ix}{y}$.

\textbf{Комплексне число (уявне число)} --- це число виду $z = x + iy; x, y \in \mathbb{R}$,
а число $i$ --- це уявна одиниця.

\textbf{Множина комплексних чисел $\mathbb{C}$} -- множина усіх дійсних чисел та комплексних чисел.

\textbf{Спряжене число} до комплексного числа --- це $\bar z = x - iy$.

\textbf{Дійсна частина} комплексного числа $z = x + iy$ --- це $Re(z) = x$.

$Re z = \dfrac{z + \bar{z}}{2}$.

\textbf{Уявна частина} комплексного числа $z = x + iy$ --- це $Im(z) = y$.

$Im z = \dfrac{z - \bar{z}}{2i}$.

Комплексне число можна подати як вектор.
В Декартовій системі координат дійсна частина зображається на осі $Ox$, а уявна на осі $Oy$, при чому за одиницю на уявній осі, ми беремо число $i$.
В полярній системі координат все те ж саме, але вектор ми подаємо не як дійсна  та уявна частина, $x$ та $y$, а як довжина вектора $\rho$ і кут його повороту $\phi$, відносно додатнього напрямку осі $Ox$.

Нехай $x, y, \rho, \phi \in \mathbb{R}$, тоді:

\textbf{Алгебраїчна форма} комплексного числа --- $z = x + iy$.

\textbf{Показникова форма} комплексного числа --- $z = |z|e^{i \phi}$.

\textbf{Тригонометрична форма} комплексного числа --- $z = |z| (\cos \phi + i \sin \phi)$.

\textbf{Модуль} комплексного числа --- $|z| = \rho = \sqrt{x^2 + y^2}$.

\textbf{Формула Ейлера(Ойлера)} --- $s^{i \phi} = \cos \phi + i \sin \phi$.

Аргумент комплексного числа --- кут $\phi$.

$Arg z$ --- множина значень аргумента $z$.

$Arg z =  \arg z + 2 \pi k, k \in \mathbb{Z}$.

Головне значення аргумента комплексного числа ---$\arg z =  \phi$, $\phi \in ( - \pi; \pi ]$.

$$\text{arg}z = \left\{ \begin{array}{lcr}
\arctan{\frac{y}{x}},       & x>0,      & (1, 4 \text{ чверті}) \\
\arctan{\frac{y}{x}} + \pi, & x<0, y>0, & (2 \text{ чверть}) \\
\arctan{\frac{y}{x}} - \pi, & x<0, y<0, & (3 \text{ чверть})
\end{array}\right. $$

$\arg z$ визначений для $z \neq 0$, а для випадків коли $x$ чи $y$ рівні 0, дивишся куди направлений вектор вручну.

\underline{Корінь комплексного числа} ---

$$\sqrt[n]{z} = \sqrt[n]{|z|}(\cos\dfrac{\phi+2\pi k}{n} + i\sin\dfrac{\phi+2\pi k}{n}), k=0,1,...,n-1.$$



\section{Послідовності комплексних чисел}

\textbf{Послідовність комплексних чисел} -- це комплекснозначна функція натурального аргумента $f(n) = z_n, n\in\mathbb{N}, z_n\in\mathbb{C}$.

$\{z_n\}$ -- послідовність.

\textbf{Ліміт послідовності комплексних чисел}:

$\lim\limits_{n\rightarrow\infty} z_n = z_0 \Leftrightarrow \forall\varepsilon > 0 ~ \exists N = N ( \varepsilon ) \in\mathbb{N} : \forall n \geqslant N(\varepsilon) ~ |z_n-z_0|<\varepsilon$.

$\{z_n\}$ -- \textbf{обмежена}, якщо 

$\exists M>0 ~ \forall n \in \mathbb{N} ~ |z_n|<M$.

\begin{theorem}
$$
\text{Нехай: } \begin{array}{l}
z_n=x_n+iy_n; \\
z_0=x_0+iy_0;
\end{array}, 
(x_n,y_n,x_0,y_0 \in \mathbb{R})
$$

$$
\text{Тоді: }\lim\limits_{n\rightarrow\infty} z_n = z_0 \Leftrightarrow 
\left\{ \begin{array}{l}
\lim\limits_{n\rightarrow\infty} x_n = x_0 \\
\lim\limits_{n\rightarrow\infty} y_n = y_0
\end{array}\right.
$$
\end{theorem}
\begin{proof}
Очевидно.
\end{proof}

\begin{claim}
$\lim\limits_{n\rightarrow\infty} z_n = z_0 \Leftrightarrow \lim\limits_{n\rightarrow\infty} |z_n-z_0| = 0$
\end{claim}

\begin{theorem}
Якщо $\lim\limits_{n\rightarrow\infty} z_n = z_0$, то $\lim\limits_{n\rightarrow\infty} |z_n| = |z_0|$
\end{theorem}

\begin{theorem}
$$ 
\left\{ \begin{array}{l}
\lim\limits_{n\rightarrow\infty} |z_n| = |z_0| \\
\lim\limits_{n\rightarrow\infty} \varphi_n = \varphi_0
\end{array}\right.
\Rightarrow \lim\limits_{n\rightarrow\infty} z_n = z_0
$$
\end{theorem}


\section{Комплексні числа. основні визначення}

\textbf{Уявна одиниця} --- це спеціальне число в математиці з незвичайними властивостями.

Властивість уявної одиниці $i^2 = -1$.

Якщо ми маємо щось схоже на $\dfrac{x}{iy}$, то, $\dfrac{x}{iy} = - \dfrac{ix}{y}$.

\textbf{Комплексне число (уявне число)} --- це число виду $z = x + iy; x, y \in \mathbb{R}$,
а число $i$ --- це уявна одиниця.

\textbf{Множина комплексних чисел $\mathbb{C}$} -- множина усіх дійсних чисел та комплексних чисел.

\textbf{Спряжене число} до комплексного числа --- це $\bar z = x - iy$.

\textbf{Дійсна частина} комплексного числа $z = x + iy$ --- це $Re(z) = x$.

$Re z = \dfrac{z + \bar{z}}{2}$.

\textbf{Уявна частина} комплексного числа $z = x + iy$ --- це $Im(z) = y$.

$Im z = \dfrac{z - \bar{z}}{2i}$.

Комплексне число можна подати як вектор.
В Декартовій системі координат дійсна частина зображається на осі $Ox$, а уявна на осі $Oy$, при чому за одиницю на уявній осі, ми беремо число $i$.
В полярній системі координат все те ж саме, але вектор ми подаємо не як дійсна  та уявна частина, $x$ та $y$, а як довжина вектора $\rho$ і кут його повороту $\phi$, відносно додатнього напрямку осі $Ox$.

Нехай $x, y, \rho, \phi \in \mathbb{R}$, тоді:

\textbf{Алгебраїчна форма} комплексного числа --- $z = x + iy$.

\textbf{Показникова форма} комплексного числа --- $z = |z|e^{i \phi}$.

\textbf{Тригонометрична форма} комплексного числа --- $z = |z| (\cos \phi + i \sin \phi)$.

\textbf{Модуль} комплексного числа --- $|z| = \rho = \sqrt{x^2 + y^2}$.

\textbf{Формула Ейлера(Ойлера)} --- $s^{i \phi} = \cos \phi + i \sin \phi$.

Аргумент комплексного числа --- кут $\phi$.

$Arg z$ --- множина значень аргумента $z$.

$Arg z =  \arg z + 2 \pi k, k \in \mathbb{Z}$.

Головне значення аргумента комплексного числа ---$\arg z =  \phi$, $\phi \in ( - \pi; \pi ]$.

$$\text{arg}z = \left\{ \begin{array}{lcr}
\arctan{\frac{y}{x}},       & x>0,      & (1, 4 \text{ чверті}) \\
\arctan{\frac{y}{x}} + \pi, & x<0, y>0, & (2 \text{ чверть}) \\
\arctan{\frac{y}{x}} - \pi, & x<0, y<0, & (3 \text{ чверть})
\end{array}\right. $$

$\arg z$ визначений для $z \neq 0$, а для випадків коли $x$ чи $y$ рівні 0, дивишся куди направлений вектор вручну.

\underline{Корінь комплексного числа} -- $\sqrt[n]{z} = \sqrt[n]{|z|}(\cos\dfrac{\phi+2\pi k}{n} + i\sin\dfrac{\phi+2\pi k}{n}), k=0,1,...,n-1$.



\section{Послідовності комплексних чисел}

\textbf{Послідовність комплексних чисел} -- це комплекснозначна функція натурального аргумента $f(n) = z_n, n\in\mathbb{N}, z_n\in\mathbb{C}$.

$\{z_n\}$ -- послідовність.

\textbf{Ліміт послідовності комплексних чисел}:

$\lim\limits_{n\rightarrow\infty} z_n = z_0 \Leftrightarrow \forall\varepsilon > 0 ~ \exists N = N ( \varepsilon ) \in\mathbb{N} : \forall n \geqslant N(\varepsilon) ~ |z_n-z_0|<\varepsilon$.

$\{z_n\}$ -- \textbf{обмежена}, якщо 

$\exists M>0 ~ \forall n \in \mathbb{N} ~ |z_n|<M$.

\begin{theorem}
$$
\text{Нехай: } \begin{array}{l}
z_n=x_n+iy_n; \\
z_0=x_0+iy_0;
\end{array}, 
(x_n,y_n,x_0,y_0 \in \mathbb{R})
$$

$$
\text{Тоді: }\lim\limits_{n\rightarrow\infty} z_n = z_0 \Leftrightarrow 
\left\{ \begin{array}{l}
\lim\limits_{n\rightarrow\infty} x_n = x_0 \\
\lim\limits_{n\rightarrow\infty} y_n = y_0
\end{array}\right.
$$
\end{theorem}
\begin{proof}
Очевидно.
\end{proof}

\begin{claim}
$\lim\limits_{n\rightarrow\infty} z_n = z_0 \Leftrightarrow \lim\limits_{n\rightarrow\infty} |z_n-z_0| = 0$
\end{claim}

\begin{theorem}
Якщо $\lim\limits_{n\rightarrow\infty} z_n = z_0$, то $\lim\limits_{n\rightarrow\infty} |z_n| = |z_0|$
\end{theorem}

\begin{theorem}
$$ 
\left\{ \begin{array}{l}
\lim\limits_{n\rightarrow\infty} |z_n| = |z_0| \\
\lim\limits_{n\rightarrow\infty} \varphi_n = \varphi_0
\end{array}\right.
\Rightarrow \lim\limits_{n\rightarrow\infty} z_n = z_0
$$
\end{theorem}
