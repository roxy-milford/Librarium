\chapter[Метричнi простори. Повнота i сепарабельнiсть]{Лекцiя 1. Метричнi простори. Повнота i сепарабельнiсть}



\textbf{Функцiональний аналiз}

поєднує i узагальнює методи математичного аналiзу та лiнiйної алгебри для
дослiдження складних просторiв, якi переважно складаються з функцiй.

\begin{itemize}
    \item I. Фредгольм, 1900: $f(x) + \int_{a}^{b} K(x, y) f(y) dy = g(x)$.
    \item А. Лебег, 1902: визначення iнтегралу $\int_{X} f d \mu$ по абстрактнiй
        множинi $X$ здовiльною мiрою $\mu$.
    \item Д. Гiльберт, 1906: визначення гiльбертового простору та розвиток
        спектральної теорiї.
    \item М. Фреше, 1906: метричнi простори.
    \item С. Банах, 1930: теорiя лiнiйних операцiй.
\end{itemize}

Результати та методи функцiонального аналiзу широко застосовуються в
лiнiйному програмуваннi, задачах оптимiзацiї, математичнiй фiзицi, теорiї
перетворення Фур’є, квантовiй фiзицi, теорiї випадкових процесiв тощо.

\textbf{Теми}
\begin{itemize}
    \item Метричнi простори.
    \item Нормованi простори.
    \item Iнтегрування.
    \item Лiнiйнi оператори.
\end{itemize}

\textbf{Лiтература}
\begin{itemize}
    \item Березанський Ю. М., Ус Г. Ф., Шефтель З. Г. Функцiональний аналiз:
        курс лекцiй.
    \item Кадець В.М. Курс функцiонального аналiзу та теорiї мiри.
    \item A.N. Kolmogorov, S.V. Fomin. Elements of the Theory of Functions and
        Functional Analysis.
    \item A.A. Kirillov, A.D. Gvishiani. Theorems and Problems in Functional Analysis.
\end{itemize}

\begin{definition}[Метричний простір, метрика]
    Метричним простором називається пара $(X, d)$, де $X$ ---
    множина, $d : X \times X \rightarrow R$, такi що
    \begin{itemize}
        \item $d(x, y) \geqslant 0; d(x, y) = 0 \Leftrightarrow x = y$;
        \item $d(x, y) = d(y, x)$;
        \item $d(x, z) \leqslant d(x, y) + d(y, z)$ (нерiвнiсть трикутника).
    \end{itemize}
    $d$ --- метрика (вiдстань) на $X$. Елементи $X$ називають точками.
\end{definition}

\begin{example}[дискретний простiр]
    Множина $X$ довiльна.
    $$d(x, y) = \left\{ \begin{array}{ll}
        1 & x \neq y \\
        0 & x = y \\
    \end{array} \right.$$
\end{example}

\begin{example}[Евклiдiв простiр]
    $X = R^n$ або $C^n$
    $$d(x, y) = \sqrt{\sum\limits_{k=1}^n |x_k - y_k|^2}.$$
\end{example}

\begin{example}
    $1 \leqslant p < \infty$; $X = R^n$ або $C^n$
    $$d_p(x, y) =  \left( \sqrt{\sum\limits_{k=1}^{n}} |x_k - y_k|^p \right)^{\frac{1}{p}}$$
\end{example}

\begin{example}
    $X = R^n$ або $C^n$
    $$d_{\infty}(x, y) = \max\limits_{1 \leqslant k \leqslant n} |x_k - y_k|.$$
\end{example}

\begin{claim}
\end{claim}
Нехай $a$, $b \geqslant 0$, $1 < p < \infty$, $q = \frac{p}{f-1}$ (так,
що $\frac{1}{p} + \frac{1}{q} = 1$. Тодi
$$ab \leqslant \frac{a^p}{p} + \frac{b^q}{q}$$
\begin{proof}
    Функцiя $x \mapsto \ln x$ угнута. Отже
    $$ab = \exp(\ln a + \ln b) = \exp \left( \frac{1}{p} \ln a^p + \frac{1}{q} \ln b^q \right)
    \leqslant \exp \left( \ln \left( \frac{1}{p} a^p + \frac{1}{q} b^q \right) \right)
    = \frac{1}{p} a^p + \frac{1}{q} b^q.$$
\end{proof}

\begin{corollary}[нерiвнiсть Гельдера]
    Нехай $a_1, ..., a_n$, $b_1, ..., b_n$ $\in \mathbb{C}$, $1 < p < \infty$,
    $q = \frac{p}{p-1}$. Тодi
    $$\sum\limits_{k=1}^{n} |a_k b_k|
    \leqslant \left( \sum\limits_{k=1}^{n} |a_k|^p \right)^{\frac{1}{p}} \left( \sum\limits_{k=1}^{n} |b_k|^q \right)^{\frac{1}{q}}.$$
\end{corollary}
\begin{proof}
    Нехай $\alpha_k = \dfrac{|a_k|}{\left( \sum\limits_{j=1}^{n} |a_j|^p \right)^{\frac{1}{p}}}$,
    $\beta_k = \dfrac{|b_k|}{\left( \sum\limits_{j=1}^{n} |b_j|^p \right)^{\frac{1}{p}}}$.
    Тодi
    $$\sum\limits_{k=1}^n \alpha_k \beta_k \leqslant
    \dfrac{1}{p} \sum\limits_{k=1}^n \alpha_k^p + \dfrac{1}{q} \sum\limits_{k=1}^n \beta_k^q
    = 1.$$
    
    Отже,
    $$\sum\limits_{k=1}^n |a_k b_k| \leqslant
    \left( \sum\limits_{k=1}^n |a_k|^p\right)^{\dfrac{1}{p}} + \left( \sum\limits_{k=1}^n |b_k|^q\right)^{\dfrac{1}{q}}.$$
\end{proof}

\begin{corollary}[нерiвнiсть Мiнковського]
    Нехай $a_1, ..., a_n, b_1, ..., b_n \in \mathbb{C}$, $1 \leqslant p < \infty$. Тодi
    $$\left( \sum\limits_{k=1}^n |a_k + b_k|^p \right)^{\frac{1}{p}} \leqslant
    \left( \sum\limits_{k=1}^n |a_k|^p \right)^{\frac{1}{p}} + \left( \sum\limits_{k=1}^n |b_k|^p \right)^{\frac{1}{p}}.$$
\end{corollary}
\begin{proof}
    Випадок $p = 1$ очевидний. Нехай $1 < p < \infty$, $q = \dfrac{p}{p-1}$.
    
    $$\begin{array}{rcl}
        \sum\limits_{k=1}^{n} |a_k + b_k|^p
        & \leqslant & \sum\limits_{k=1}^{n} |a_k| |a_k + b_k|^{p-1} + \sum\limits_{k=1}^{n} |b_k| |a_k + b_k|^{p-1}  \\
        & \leqslant &
        \left(
            \left( \sum\limits_{k=1}^n |a_k|^p \right)^{\frac{1}{p}}
            + \left( \sum\limits_{k=1}^n |b_k|^p \right)^{\frac{1}{p}}
        \right) \left( \sum\limits_{k=1}^n |a_k + b_k|^p \right)^{\frac{1}{q}} \\
    \end{array}$$
    
    Отже,
    $$\left( \sum\limits_{k=1}^n |a_k + b_k|^p \right)^{\frac{1}{p}} \leqslant
    \left( \sum\limits_{k=1}^n |a_k|^p \right)^{\frac{1}{p}} + \left( \sum\limits_{k=1}^n |b_k|^p \right)^{\frac{1}{p}}.$$
\end{proof}


\begin{theorem}
    Для кожного $p \in [1, \infty)$
    $$d_p(x, y) = \left( \sum\limits_{k=1}^{n} |x_k - y_k|^p \right)^{\frac{1}{p}}$$
    
    є метрикою на $\mathbb{C}^n$.
\end{theorem}
\begin{proof}
    Нерiвнiсть трикутника випливає з нерiвностi Мiнковського:
    $$\begin{array}{rcl}
        d_p(x, z)
        & = & \left( \sum\limits_{k=1}^{n} |x_k - z_k|^p \right)^{\frac{1}{p}}  \\
        & = & \left( \sum\limits_{k=1}^{n} |(x_k - y_k) + (y_k - z_k)|^p \right)^{\frac{1}{p}}  \\
        & \leqslant & \left( \sum\limits_{k=1}^{n} |(x_k - y_k)|^p \right)^{\frac{1}{p}}
            + \left( \sum\limits_{k=1}^{n} |(y_k - z_k)|^p \right)^{\frac{1}{p}}  \\
        & = & d_p(x, y) + d_p(y, z)  \\
    \end{array}$$
\end{proof}

\begin{example}
    $X = \mathbb{R}^n$ або $\mathbb{C}^n$, $0 < p < 1$.
    $$d_p(x, y)
    = \sum\limits_{k=1}^n |x_k - y_k|^p$$
    
    є метрикою на $X$.
    
    \begin{proof}
        $f(x) = 1 + x^p - (1 + x)^p$. При $x > 0$,
        
        $$f'(x)
        = p x^{p-1} - p(1 + x)^{p-1}
        = p \left(\dfrac{1}{x^{1-p}} - \dfrac{1}{(1+x)^{1-x}} \right)
        > 0.$$
        
        Отже,
        $$(1 + x)^p \leqslant 1 + x^p;$$
        $$(a + b)^p \leqslant a^p + b^p;$$
        $$d_p(x, z) \leqslant d_p(x, y) + d_p(y, z).$$
    \end{proof}
\end{example}

\begin{example}
    $1 \leqslant p \leqslant \infty$, $l_p$ --- простiр всiх
    послiдовностей $x = (x_1, x_2, ...)$, для яких
    $\sum\limits_{k=1}^{\infty} |x_k|^p < \infty$.
    
    $$d_p(x, y) = \left( \sum\limits_{k=1}^{\infty} |x_k - y_k|^p \right)^{\frac{1}{p}}$$
    
    є метрикою на $l_p$.
\end{example}

\begin{example}
    $0 < p < 1$, $l_p$ --- простiр всiх послiдовностей $x = (x1, x2, ...)$, для яких
    $\sum\limits_{k=1}^{\infty} |x_k|^p < \infty$.
    
    $$d_p(x, y) = \left( \sum\limits_{k=1}^{\infty} |x_k - y_k|^p \right).$$
    
    є метрикою на $l_p$.
\end{example}

\begin{example}
    $C[a, b]$ – простiр всiх неперервних функцiй $f: [a, b] \rightarrow \mathbb{R}$.
    $$d(f, g) = \max\limits_{t \in [a,b]} |f(t) - g(t)|.$$
    є метрикою на $C[a, b]$.
\end{example}

\begin{example}
    $C[a, b]$ --- простiр всiх неперервних функцiй $f: [a, b] \rightarrow \mathbb{R}$.
    $$d(f, g) = \int\limits_{a}^{b} |f(t) - g(t)| dt.$$

    є метрикою на $C[a, b]$.
\end{example}

\begin{example}
    $C_b(\mathbb{R})$ --- простiр всiх неперервних i обмежених функцiй
    $f: \mathbb{R} \rightarrow \mathbb{R}$.
    $$d(f, g) = \sup\limits_{t \in \mathbb{R}} |f(t) - g(t)|.$$
\end{example}

\begin{claim}[Пiдпростiр]
    Якщо $(X, d)$ --- метричний простiр i $Y \subset X$, то звуження $d$ на
    $Y \times Y$ є метрикою на $Y$. $Y$ --- пiдпростiр простору $X$.
\end{claim}

\begin{definition}[Збіжність послідовності до точки]
    Послiдовнiсть $x_1$, $x_2$, ... точок метричного простору $(X, d)$
    збiгається до точки $x \in X$, якщо
    $$\lim\limits_{n \rightarrow \infty} d(x_n, x) = 0.$$
    
    Еквiвалентно:
    $$\forall \varepsilon > 0 \quad
    \exists N \geqslant 1: \quad
    \forall n \geqslant N d(x_n, x) <
    \varepsilon.$$
\end{definition}

$x = \lim\limits_{n \rightarrow \infty} x_n$ --- границя послiдовностi.

\begin{theorem}
    Границя збiжної послiдовностi єдина.
\end{theorem}
\begin{proof}
    Нехай $x$ та $y$ --- границi збiжної послiдовностi $(x_n)_{n \geqslant 1}$.
    Для довiльного $\varepsilon > 0$ iснує $n: d(x_n, x) < \varepsilon/2$, $d(x_n, y) < \varepsilon/2$.
    Тодi $d(x, y) \leqslant d(x_n, x) + d(x_n, y) < \varepsilon$. $d(x, y) = 0$ i $x = y$.
\end{proof}

\begin{definition}[Неперервність відображення в точці]   
    Нехай $(X, d)$, $(Y, \rho)$ --- метричнi простори, $f : X \rightarrow Y$.
    Вiдображення $f$ неперервне в точцi $x_0 \in X$, якщо для довiльного
    $\varepsilon > 0$ iснує $\sigma > 0$ таке що
    $$x \in X, d(x, x_0) < \sigma \Rightarrow \rho(f(x), f(x_0)) < \varepsilon.$$
    
    Вiдображення $f$ неперервне, якщо $f$ неперервне в кожнiй точцi $x_0 \in X$.
\end{definition}

\begin{problem}
Вiдображення $f: X \rightarrow Y$ неперервне в точцi $x_0 \in X$ тодi i тiльки тодi,
коли для довiльної послiдовностi $x_1$, $x_2$, ... точок $X$, яка збiгається до $x_0$,
послiдовнiсть $f(x_1)$, $f(x_2)$, ... збiгається до $f(x_0)$ в $Y$.
\end{problem}

\begin{definition}[Гомеоморфiзм]
    Вiдображення $f: X \rightarrow Y$ називається гомеоморфiзмом, якщо
    $f$ бiєктивне, $f$ та $f^{-1}$ неперервнi.
\end{definition}

\begin{definition}[Ізометрiя]
    Вiдображення $f: X \rightarrow Y$ називається iзометрiєю, якщо
    $$\rho(f(x), f(x')) = d(x, x').$$
\end{definition}

\begin{problem}
    Iзометрiя неперервна. Якщо $f$ iзометрiя та бiєкцiя, то $f$ – гомеоморфiзм.
\end{problem}

\begin{example}
    Нехай $X = \mathbb{R}$, $d(x, x') = |x - x'|$, $Y = (-1, 1)$ – пiдпростiр $X$.
    Тодi $f(x) = \dfrac{x}{1+|x|}$ --- гомеоморфiзм мiж $X$ та $Y$.
\end{example}

\begin{definition}[Відкрита куля]
    Вiдкритою кулею з центром в точцi $x_0 \in X$ i радiусом $r > 0$ називається
    множина
    $$B(x_0, r) = \{x \in X: d(x, x_0) < r\}.$$
\end{definition}

\begin{definition}[Замкнена куля]
    Замкненою кулею з центром в точцi $x_0 \in X$ i радiусом $r > 0$ називається
    множина
    $$B(x_0, r) = \{x \in X: d(x, x_0) \leqslant r\}.$$
\end{definition}

\begin{definition}[Сфера]
    Сферою з центром в точцi $x_0 \in X$ i радiусом $r > 0$ називається множина
    $$S(x_0, r) = \{x \in X: d(x, x_0) = r\}.$$
\end{definition}

\begin{definition}[Вiдкрита множина]
    Множина $G \subset X$ вiдкрита, якщо 
    $$\forall x \in  G \quad \exists r > 0: B(x, r) \subset G.$$
\end{definition}

\begin{definition}[Дотична до множини]
    Точка $x \in X$ дотична до множини $A \subset X$, якщо
    $$\forall r > 0 \quad B(x, r) \cap A \neq \varnothing.$$
\end{definition}

\begin{definition}[Замикання]
    Замикання $\overline{A}$ --- це сукупнiсть всiх дотичних точок множини $A$.
\end{definition}

\begin{definition}[Замкнена множина]
    Множина $A$ замкнена, якщо вона мiстить всi свої дотичнi точки, тобто
    $\overline{A} = A$.
\end{definition}


Теорема
\begin{itemize}
    \item $A \subset \overline{A}$
    \item $A \subset C \Rightarrow \overline{A} \subset \overline{C}.$
    \item $\overline{A \cup C} = \overline{A} \cup \overline{C}$
    \item $\overline{\overline{A}} = \overline{A}$
\end{itemize}
\begin{proof}
    (1) Якщо $x \in A$, то $x \in B(x, r) \cap A \neq \varnothing$.
    Отже, $x \in \overline{A}$.
    
    (2) Якщо $A \subset C$ i $x \in A$, то для довiльного $r > 0 B(x, r) \cap A \neq \varnothing$.
    $$B(x, r) \cap A \subset B(x, r) \cap C \Rightarrow B(x, r) \cap C \neq \varnothing, x \in \overline{C}.$$
    
    (3) Включення $A \cup C \subset A \cup C$ випливає з (2).
    Нехай $x \in A \cup C$. Якщо $x \not\in \overline{A}$,
    то iснує $r_0 > 0: B(x, r_0) \cap A = \neq$.
    Для довiльного $r > 0 B(x, \min(r, r_0)) \cap (A \cup C) \neq \varnothing$.
    Отже, $B(x, \min(r, r_0)) \cap C \neq \varnothing$.
    $B(x, \min(r, r_0)) \subset B(x, r)$ i $B(x, r) \cap C \neq \varnothing$.
    З припущень $x \in \overline{A \cup C}, x \not\in \overline{A}$
    випливає $x \in \overline{C}$. Отже, $\overline{A \cup C} \subset \overline{A} \cup \overline{C}$.
    
    (4) Враховуючи (1) достатньо довести, що $\overline{\overline{A}} \subset \overline{A}$.
    Нехай $x \in \overline{\overline{A}}$ i $r > 0$.
    Вiдкрита куля $B(x, r)$ мiстить точку $y \in \overline{A}$. Нехай $\varepsilon = r - d(x, y)(> 0)$.
    Покажемо, що $B(y, \varepsilon) \subset B(x, r)$. Дiйсно, якщо $z \in B(y, \varepsilon)$, то
    $$d(z, x) \leqslant d(z, y) + d(y, x) < \varepsilon + d(y, x) = r, z \in B(x, r).$$

    Так як $y \in \overline{A}$, то $B(y, \varepsilon) \cap A \neq \varnothing$.
    Так як $B(y, \varepsilon) \subset B(x, r)$, то
    $B(x, r) \cap A \neq \varnothing$ i $x \in \overline{A}$.
\end{proof}

\begin{corollary}
    Замикання $\overline{A}$ будь-якої множини $A$ замкнене.
\end{corollary}

\begin{theorem}
    $x \in \overline{A}$ тодi i тiльки тодi, коли $x$ є границею деякої послiдовностi точок
    множини $A$.
\end{theorem}
\begin{proof}
    Нехай $x = \lim\limits_{n \rightarrow \infty} x_n, x_1, x_2, ... \in A$.
    Для довiльного $r > 0$ iснує $n$ таке що $d(x_n, x) < r$.
    Тобто $x_n \in B(x, r) \cap A \neq \varepsilon$. $x \in \overline{A}$.
    Навпаки, нехай $x \in \overline{A}$.
    Для довiльного $n \geqslant 1$ вiдкрита куля $B(x, \frac{1}{n})$ мiстить
    точку $x_n \in A$. За побудовою, $d(x, x_n) < \frac{1}{n}$.
    Отже, $x = \lim\limits_{n \rightarrow \infty} x_n$.
\end{proof}


Визначення
\begin{itemize}
    \item Множина $A$ щiльна в множинi $B$, якщо $B \subset A$.
    \item Множина $A$ всюди щiльна, якщо вона щiльна в $X$ (тобто $\overline{A} = X$).
    \item Простiр $X$ сепарабельний, якщо в ньому iснує злiченна всюди щiльна множина.
\end{itemize}

Приклади
\begin{itemize}
    \item Простiр $\mathbb{R}^n$ сепарабельний вiдносно кожної з метрик $d_p$,
    $0 < p \leqslant \infty$. В якостi всюди щiльної множини можна взяти множину
    $\mathbb{Q}^n$ точок з рацiональними координатами.
    
    \item Простiр $C[a, b]$ сепарабельний вiдносно метрики
    $d(f, g) = \max\limits_{t \in [a,b]} |f(t) - g(t)|$.
    В якостi всюди щiльної множини можна
    взяти множину многочленiв з рацiональними коефiцiєнтами (теорема
    Вайєрштрасса).
    
    \item $1 \leqslant p < \infty$. Простiр $l_p$ сепарабельний вiдносно метрики
    $d(x, y) = ( \sum\limits_{n=1}^{\infty} |x_n - y_n|^{p})^{\frac{1}{p}}$.
    В якостi всюди щiльної множини можна
    взяти множину послiдовностей з рацiональними компонентами, в яких
    тiльки скiнченна кiлькiсть компонент не рiвна нулю.
\end{itemize}

Приклади
\begin{itemize}
    \item Незлiченний дискретний простiр не сепарабельний. В дискретному
    просторi $\overline{A} = A$.
    
    \item Простiр $C_b(\mathbb{R})$ не сепарабельний вiдносно метрики
    $d(f, g) = \sup\limits_{t \in \mathbb{R}} |f(t) - g(t)|$.
    
    Для довiльної множини $A \subset \mathbb{Z}$ iснує обмежена неперервна функцiя $f_A$,
    така що
    $$f_A(n) = \left\{ \begin{array}{ll}
        1, & n \in A \\
        0, & n \not\in A \\
    \end{array} \right. .$$
    
    Якщо $A' \neq A'$, то $d(f_A, f_{A'}) \geqslant 1$. $\{B(f_A, \frac{1}{2})\}_{A \subset \mathbb{Z}}$
    --- незлiченна сукупнiсть вiдкритих куль, якi попарно не перетинаються.
\end{itemize}

\begin{definition}[Фундаментальна послiдовнiсть]
    Послiдовнiсть $(x_n)_{n \geqslant 1}$ точок метричного простору $X$ фундаментальна
    (послiдовнiсть Кошi), якщо $\lim\limits_{n,m \rightarrow \infty} d(x_n, x_m) = 0$.
    Тобто, $\forall \varepsilon > 0 \quad \exists N \geqslant 1$:
    $$n, m \geqslant N \Rightarrow d(x_n, x_m) < \varepsilon.$$
\end{definition}

\begin{theorem}
    Збiжна послiдовнiсть фундаментальна.
\end{theorem}
\begin{proof}
    Нехай $x = \lim\limits_{n \rightarrow \infty} x_n$.
    Для заданого $\varepsilon > 0$ iснує iндекс $N \geqslant 1$,
    такий що $d(x_n, x) < \frac{\varepsilon}{2}$ при $n \geqslant N$.
    Тодi для $n, m \geqslant N$ виконується
    $$d(x_n, x_m) \leqslant d(x_n, x) + d(x_m, x) < \varepsilon.$$
\end{proof}


\begin{definition}[Повний метричний простiр]
    Метричний простiр --- повний, якщо в ньому будь-яка фундаментальна
    послiдовнiсть збiгається.
\end{definition}

