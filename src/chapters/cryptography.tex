\chapter{Cryptography}
\localtableofcontents

\section{Symetrical cryptography}


\section{Asymetrical cryptography}

Some rules
\begin{enumerate}
    \item Moral: Don’t share a common n among a group of users. 
    \item Moral: Pad messages with random values before encrypting them;
        make sure m is about the same size as n.
    \item Moral: Choose a large value for key (for example for d in RSA).
    \item — Knowledge of one encryption/decryption pair of exponents for
        a given modulus enables an attacker to factor the modulus.
    \item — Knowledge of one encryption/decryption pair of exponents for
        a given modulus enables an attacker to calculate other
        encryption/decryption pairs without having to factor n.
    \item — A common modulus should not be used in a protocol using RSA in
        a communications network. (This should be obvious from the previous
        two points.)
    \item — Messages should be padded with random values to prevent attacks
        on low encryption exponents.
    \item — The decryption exponent should be large. 
\end{enumerate}


RSA Encryption
\begin{enumerate}
    \item[]\textbf{Public Key:}
    $n$ product of two primes, $p$ and $q$ ($p$ and $q$ must remain secret)
    $e$ relatively prime to $(p - 1)(q - 1)$.
    \item[]\textbf{Private Key:}
    $d^{e-1} \mod ((p - 1)(q - 1)).$
    \item[]\textbf{Encrypting:}
    $c = me \mod n.$
    \item[]\textbf{Decrypting:}
    $m = cd \mod n.$
\end{enumerate}










% \section{Text encryption}

% \section{Random generators}
% In \cite{BS15}, we can see some examples of random generators.

% In \cite{BS15} we can see, that the best real random generators are
% maden with use of real world randomnesss.

% \section{Literture overview}

% \begin{table}[!ht]
%     \centering
%     \begin{tabular}{c|c|c|c|c|c|c}
%                     & Type    & Sym & Asym & Hash & randomness & protocols\\
%         \hline\hline
%         \cite{BS15} & Glosary & Yes & Yes  & Yes  & Yes        & Yes \\
%     \end{tabular}
%     \caption{Caption}
%     \label{tab:my_label}
% \end{table}


% Book \cite{BS15} -- is general overview of begining of cryptography.


