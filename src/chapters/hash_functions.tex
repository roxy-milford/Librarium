\chapter{Hash functions in cryptography}

\section{Introduction in to cryptographic hash functions}

\begin{definition}[Hash function]
    Hash function is a relation
    $$\h: \{0, 1\}^{*} \rightarrow \{0, 1\}^{n}.$$
\end{definition}

\begin{definition}[Keyed hash function]
    a keyed hash function is a relation
    $$\h_k: \mathcal{K} \times \{0, 1\}^{*} \rightarrow \{0, 1\}^{n},$$
    where $\mathcal{K}$ is a set of parameters (keys).
\end{definition}

% \begin{definition}[One-way hash function]
%     One-way hash function is a hash function that is hard to reverse.
% \end{definition}

\begin{remark}
    Hash functions can have any string as input, but the output is of fixed size.
\end{remark}

\begin{definition}[Message Authentication Code (MAC)]
    Message Authentication Code (MAC or MAC-code) is a keyed hash function
    with a secret key.
\end{definition}

\begin{definition}[Salt]
    if the key in a keyed hash function is public, then the key is a salt.
\end{definition}

Hash functions properties:
\begin{enumerate}[noitemsep]
    \item \textbf{Output size.}
        size of output of the function that is always fixed.
    \item \textbf{Speed. (kilobytes/second)}
        Number of kilobytes processed by hash function at a second.
    \item \textbf{Uniform distribution.}
        All outputs of a hash function are uniformly distributed.
    \item \textbf{Avalanche effect.}
        The slightest change in the input of a hash function leads
        to unpredictable changes in output
    \item \textbf{Simplicity and Compactness.}
        Simple and small implementation.
    \item \textbf{Preimage-resistance.}
        For essentially all pre-specified outputs,
        it is computationally infeasible to find any input which hashes to that output,
        i.e., to find any preimage $x'$ such that $\h(x') = y$ when given any $y$ for which
        a corresponding input is not known.
    \item \textbf{2nd-preimage resistance.}
        It is computationally infeasible to find any second
        input which has the same output as any specified input, i.e., given $x$, to find
        a 2nd-preimage $x' \neq x$ such that $\h(x) = h(x')$.
    \item \textbf{Collision resistance.}
        It is computationally infeasible to find any two distinct
        inputs $x$, $x'$ which hash to the same output, i.e., such that $\h(x) = \h(x')$.
\end{enumerate}

\begin{definition}[Negligible function]
    In mathematics, a negligible function is a function  $\mu: \mathbb{N} \rightarrow \mathbb{R}$
    such that for every positive integer $c$ there exists an integer $N_c$ such that for all $x > N_c$,
    $|\mu (x)| < \frac {1}{x^{c}}.$
    
    Equivalently, we may also use the following definition. A function 
    $\mu :\mathbb {N} \to \mathbb {R}$ is negligible, if for every positive
    polynomial $poly(\cdot)$ there exists an integer $N_{poly} > 0$ such that
    for all $x > N_{poly}$
    $|\mu (x)|<{\frac {1}{\operatorname {poly} (x)}}.$
\end{definition}

For the last three properties, there is more formal description.
Let $\mu(x)$ be a negligible function. Then:
\begin{enumerate}[noitemsep]
    \item \textbf{Preimage-resistance.}
        Let have an adversary
        $A_1: \{0, 1\}^{n} \rightarrow \{0, 1\}^{\leqslant l(n)},$
        limited with time $T$. Then:
        $$P_{X, A_1}\{\h(A_1(\h(x))) = \h(x)\} \leqslant \mu(n).$$
    \item \textbf{2nd-preimage resistance.}
        Let have an adversary
        $A_2: \{0, 1\}^{\leqslant l(n)} \rightarrow \{0, 1\}^{\leqslant l(n)},$
        limited with time $T$. Then:
        $$P_{X, A_2}\{\h(A_2(x)) = \h(x)\} \leqslant \mu(n).$$
    \item \textbf{Collision resistance.}
        Let have an adversary
        $A_3: \mathcal{K} \rightarrow \{0, 1\}^{\leqslant l(n)} \times \{0, 1\}^{\leqslant l(n)} \cup \bot,$
        limited with time $T$. Then:
        $$P_{\mathcal{K}, A_3}\{A_3(k) \neq \bot\} \leqslant \mu(n).$$
\end{enumerate}

\begin{lemma}[Damgor]
    Collision resistance implies 2nd-preimage resistance of hash functions
\end{lemma}

\begin{remark}
    Collision resistance does not guarantee preimage resistance
\end{remark}

\section{Standard attacks}

The ideal hash function $h: \mu \rightarrow X$, $|X| = N$, ($N = 2^{n}$)
is indistinguishable from a random relation.

\begin{enumerate}[noitemsep]
    \item \textbf{Random preimage search. \ref{alg:Random_preimage_search}}
        \begin{algorithm2e}[ht]
            \caption{Random preimage search}
            \label{alg:Random_preimage_search}
            
            \KwInput{Hash value $h$ or $x'$, (where $h = \h(x')$).}
            \KwOutput{Preimage of $h$.}

            \While{$True$}{
                Generate random $x \in_{R} \mu$\;
                \If{$\h(x) = h$}{
                    \Return $x$\;
                }
            }
        \end{algorithm2e}
        \begin{lemma}
            Let $X$ ($|X| = N$) be a set, $x \in X$ is fixed element,
            $X_1 \subseteq_{R} X$, $|X_1| = r$. Then,
            $$P\{x \in X_1\} \geqslant 1 - e^{- \frac{r}{N}}.$$
        \end{lemma}
        \begin{corollary}
            For the probability $p$ of success,
            it is necessary to have the power of $X_1$: $r = N \ln(\frac{1}{1-p})$.
        \end{corollary}

    \item \textbf{Random search of a collision. (Birthday attack.) \ref{alg:Birthday_attack}}
        \begin{algorithm2e}[ht]
            \caption{Random search of a collision}
            \label{alg:Birthday_attack}
            
            % \KwInput{Hash value $h$ or $x'$, (where $h = \h(x')$)}
            \KwOutput{Collision.}

            \For{$i \leftarrow 1$ \KwTo $t^{\infty}$}{
                Generate random $x \in_{R} \mu$\;
                \If{$\h(x_i) = \h(x_j), j = \overline{1, i-1}$}{
                    \Return $(x_i, x_j)$\;
                }
            }

            \Return $\bot$\;
        \end{algorithm2e}
        \begin{lemma}
            Let $X$ ($|X| = N$) be a set.
            Let $X_1 \subseteq_{R} X$ random sample with return, $|X_1| = r$. Then,
            $$P\{X_1 \text{ has a collision}\} \geqslant 1 - e^{- \frac{r^2}{2N}}.$$
        \end{lemma}
        \begin{remark}
            It is a common practice to write $P \approx 1 - e^{- \frac{r^2}{2N}}$.
        \end{remark}
        \begin{corollary}
            For the probability of success $p$,
            it is necessary to have $r \approx \sqrt{2N \ln(\frac{1}{1-p})}$.
        \end{corollary}

    \item \textbf{Random search of a collision. (Yuval attack.) \ref{alg:Yuval_attack}}
        \begin{algorithm2e}[ht]
            \caption{Random search of a collision}
            \label{alg:Yuval_attack}
            
            % \KwInput{Hash value $h$ or $x'$, (where $h = \h(x')$)}
            \KwOutput{Collision.}

            Generate $r_1$ messages $\{x_i\}$\;
            Generate $r_2$ messages $\{y_j\}$\;
            Search for a collision $(x_i, y_j): \h(x_i) = \h(y_j) \vee x_i \neq x_j$;

            \Return $(x_i, y_j) \vee \bot$\;
        \end{algorithm2e}
        \begin{lemma}[Yuval]
            Let $X$ ($|X| = N$) be a set.
            Let $X_1, X_2 \subseteq_{R} X$ random samples with return,
            $|X_1| = r_1$, $|X_2| = r_2$. Then,
            $$P\{X_1 \cap x_2 \neq \varnothing\} \geqslant 1 - e^{- \frac{r_1 r_2}{N}}.$$
        \end{lemma}
        \begin{corollary}
            For the probability of success $p$, and generated first sample $X_1$,
            it is necessary to have $r_2 \approx \sqrt{\frac{N}{r_1} \ln(\frac{1}{1-p})}$.
        \end{corollary}
        \begin{remark}
            In case of $r_1 = 1$, we have a random preimage search.
        \end{remark}
        \begin{corollary}
            If $r_1 = r_2 = \beta \sqrt{N}$ we have a probability $p(\beta) = 1 - e^{- \beta^2}$.
        \end{corollary}
\end{enumerate}

\section{General iterative hash function. Construction of Merkle Damgard}

Let $m$ be a size of one block of the input message,
$w$ be a size of the hashing state,
$n$ be a size of the output.

The general iterative hash function construction can be described
in algorithm \ref{alg:General_Merkle_Damgard_construction}

\begin{algorithm2e}[ht]
    \caption{General Merkle Damgard construction}
    \label{alg:General_Merkle_Damgard_construction}
    
    \KwInput{$x$, $|x| \leqslant poly(n)$.}
    \KwOutput{$\h(x)$, $|\h(x)| = n$.}

    $\Tilde{x} = x || \pad(x)$ such that $m \mid |\Tilde{x}|$\;
    $\Tilde{x} = x_1 || x_2 || ... || x_t, \quad |x_i| = m$\;
    \tcc{Here $IV$ is an initialisation vector (fixed constant)}
    $H_0 = IV$\;
    \For{$i \leftarrow 1$ \KwTo $t$}{
        $H_i = f(H_{i-1}, x_i)$\;
    }

    \Return $g(H_t)$\;
\end{algorithm2e}

\begin{figure}[ht]
    \centering
    \includegraphics[width=0.4\linewidth]{example-image-a}
    \caption{General Merkle Damgard construction}
    \label{fig:General_Merkle_Damgard_construction}
\end{figure}

\begin{remark}
    Sometimes it is useful to write $\h(IV, x)$ to see a relation to the
    initialisation vector.
\end{remark}

\section{Classical Merkle Damgard construction}

let $w = n$, $g(x) = x$ ($\h(x) = H_t$).

\begin{attack}[Attack: Collision with a free prefix]
    Find different $(IV, x)$ and $(IV', x')$ such that $\h(IV, x) = \h(IV', x')$.

    $\h(Iv, x || \pad(x) || y) = \h(H, y)$ -- collision with a free prefix.
\end{attack}

\begin{solution}[Padding requirements to prevent attack]
    Paddings must meet Belare conditions.
    \begin{enumerate}[noitemsep]
        \item $x$ is a prefix of a $\Tilde{x}$. (padding appends at the end).
        \item $|x| = |y| \Rightarrow |\pad(x)| = |\pad(y)|$.
        \item $|x| \neq |y| \Rightarrow $ last blocks of $\pad(x)$ and $\pad(y)$
            must be different.
    \end{enumerate}
\end{solution}

\begin{definition}[MD-padding]
    MD-padding is $\pad(x) = 10...0||\langle length(x) \rangle$.
    Where the first ``1'' is necessity and ``0'' are appended after ``1'' to
    fill the block, so length can fill the whole one.
\end{definition}

MD-construction strengthening:
\begin{enumerate}[noitemsep]
    \item \textbf{MD-strengthening}: use MD-padding.
    \item \textbf{3C-strengthening}: append to the message block
        $x_{t+1} = \mu(H_1 \oplus H_2 \oplus ... \oplus H_t)$,
        $\mu: V_w \rightarrow V_m$.
    \item \textbf{``ГОСТ''-strengthening}: $x_{t+1} = (x_1 + x_2 + ... + x_t) \mod 2^m$.
        using in ``ГОСТ 34.11-95'' and in a ``Стрибог''.
        \begin{remark}[2007]
            Gmuravaran and Kelli showed that linear sums aren't helpful.
        \end{remark}
\end{enumerate}

\begin{theorem}[Theorem about security of MD-hashes by Damgar, Merkle]\label{theo:MD_hashes_security}
    Let $\h$ is a hash function that is a Merkle Damgard construction
    $(g(x) = x)$ with MD-strenthening.

    If $f$ is collision resistant, than $\h$ is collision resistant.
\end{theorem}
\begin{proof}
    Let exists efective algorithm of collision search for $\h$, $x \neq y$,
    $\h(x) = \h(y)$. Then:
    $$\Tilde{x} = x_1 || x_2 || ... || x_t, \Tilde{y} = y_1 || y_2 || ... || y_t.$$
    $$H_t = H_s' \Rightarrow f(H_{t-1}, x_t) = f(H_{s-1}, y _s) \Rightarrow$$
    There is a collision  or $H_{t-1} = H(s-1)$ and $x_t = y_s \Rightarrow t = s$
    (because the last block is an encoded length).
    $\Rightarrow f(H_{t-2}, x_{t-1}) = f(H_{t-2}', y_{t-1}) \Rightarrow$
    Or there is a collision or all inputs are the same.
    $$x \neq y, t = s \Rightarrow \exists i: \quad x_i \neq y_i$$
    Where inputs are different, so the assumption is wrong.
\end{proof}

\begin{remark}
    \begin{enumerate}
        \item Theorem \ref{theo:MD_hashes_security} creates a sufficient but not
            necessary condition. There exists examples of collision resistant
            hash functions $\h$ but not collision resistant functions $f$.
        \item Instead of MD-strengthening, it is possible to use Belare conditions
            for paddings
    \end{enumerate}
\end{remark}

\begin{theorem}[Lai, Messi]
    Let $\h$ be an iterative hash function with MD-strengthening.
    Let number of messages with more than $2x$ blocks is not limited.
    If for almost everyone vector $H \in V_n$ such that $(H', M'): f(H', M') = H$
    requires $2^s$ operations, then exists attack on second preimage for $\h$ with
    complexity $2 \cdot 2^{\frac{m + s}{2}}$
\end{theorem}
\begin{proof}
    Let $x$ be a given text. Then exists $y_1$ and $y_2$ such that:
    $$\h(x) = \h(x_1 || x_2 || x_3 || ...) = \h(y_1 || y_2 || x_3 || ...).$$
    1) For $2^{\frac{m + s}{2}}$ random $y_1 \in V_m$ compute $G_1 = f(H_0, y_1)$.
    2) For $2^{\frac{m - s}{2}}$ random $G_1'\in V_w$ search preimage
        $y_2: f(G', y_2) = H_2$.
    3) By the lemma of Yuval, $\exists G_1 = G_1'$ with probability
        $p(1) = 1 - e^{-1} \approx 0.63$. (more than $\frac{1}{2}, what is good$).
        (search of random vectors isn't hard, because of the model assumptions).
\end{proof}

\begin{remark}
    We proved: if $f$ is first preimage search resistant, then $\h$ isn't resistant.
    We didn't prove: if $f$ resistant, then $\h$ isn't resistant. 
\end{remark}

It was proved that MD-construction does not save property of first preimage search
resistance.


\begin{table}[ht]
    \centering
    \begin{tabular}{c|c}
        Pre & Preimage search resistance \\
        Sec & Second preimage search resistance \\
        Coll & Collision resistance \\
    \end{tabular}
    \caption{Hash function properties.}
    \label{tab:my_label}
\end{table}



\begin{table}[ht]
    \centering
    \begin{tabular}{c|c|c}
        Attack & probability & range \\ \hline\hline
        Pre   & $p \geqslant 1 - e^{-\frac{r}{N}}$ & $r \approx N \ln(\frac{1}{1 - p})$\\
        Coll  & $p \geqslant 1 - e^{-\frac{r^2}{2N}}$ & $r \approx \sqrt{2 N \ln(\frac{1}{1 - p})}$\\
        Yuval & $p \geqslant 1 - e^{-\frac{r_1 r_2}{N}}$ & $r \approx \frac{n}{r_1} \ln(\frac{1}{1 - p})$
    \end{tabular}
    \caption{Reference: attacks on a cryptographic hash functions}
    \label{tab:reference_attacks}
\end{table}

\section{Approaches for building compressive reflections}
\begin{enumerate}[noitemsep]
    \item With a use of block ciphers.
    \item With a use of stream ciphers.
    \item With a use of arithmetic, algebraic or other operations (arithmetic hashing,
        modular hashing), usage of RSA and DSA as a black box (MESH-1, MESH-2).
    \item based on NP-hard problems.
    \item other exotics (game of life, dynamic chaos, neuron links).
\end{enumerate}

\subsubsection*{building compressive reflections based on block ciphers}

1987: the first hash function (Rabin): $f(H, x) = DES_x(H)$. It is bad.

There are better candidates.
\begin{remark}
    Those functions are considered as hard reversible functions, but there is no
    proof of it. It isn't clear what requirements we need for a cipher $E$ to
    consider function $f$ as a good function.
\end{remark}

\begin{enumerate}[noitemsep]
    \item Matesh-Meer-Osis function (1985): $f(H, x) = E_{K(H)}(x) \oplus x$.
    \item Devis-Meer function (1984): $f(H, x) = E_{K(H)}(H) \oplus H$.
    \item Matesh-Meer-Osis function (1989): $f(H, x) = E_{K(H)}(x) \oplus H \oplus x$.
    \item Prenel, Govaerts, Vandervalle (1993):
        $f(H, x) = E_A(B) \oplus C, \quad A, B, C \in \{const, H, x, H \oplus x\}$.
        There are 64 compressive functions (PGV functions), but only 12 are resistant
        to appropriate attacks. Among those, there are 8, with non-movable points.
        Also, those functions can be divided on two equality classes, divided by
        functions MMO and MP.
    \item Black, Rogevai, Shrineton (2002): if $E$ is ideal oracul, then 12 functions
        of PGV are reliable.
    \item Black (2005): There is a way to make any PGV function unreliable.
\end{enumerate}
Here, $K(x)$ returns key from a vector.

\subsubsection*{pipe constructions}

\begin{definition}[Wide-pipe construction]
    By Stefan Luks (2004). Idea: make $w \geqslant 2n$. Then:
    \begin{enumerate}
        \item Etalon preimage search complexity $h = O(2^n)$.
        \item Etalon collision search complexity $f = O(2^{\frac{w}{1}}) = O(2^n)$.
    \end{enumerate}
    So attack of long messages is ineffective.
\end{definition}

\begin{definition}[Narrow-pipe construction]
    If $w = n$ we have narrow-pipe.
\end{definition}

\begin{definition}[Fast-pipe construction]
    Fast-pipe construction (2010). Idea: usage of tweak.
    $f$ is a classical function $f: V_w \times V_m \rightarrow V_w$, with usage of tweak.
    $$\varphi: V_w \times V_m \times \{0, 1\} \rightarrow V_w$$
    $$f(H, x) = (\varphi(H, x, 0), \varphi(H, x, 1))$$
\end{definition}

\section{Randomised hashing}

RMX scheme, 2006, Halevi, Kravchek.

\begin{algorithm2e}[ht]
    \caption{Randomised hashing}
    % \label{alg:default}
    
    \KwInput{Some message $x$.}
    \KwOutput{hash of message $x$, $\h(x)$, and random salt used in the algorithm.}

    Generate random salt $r \in_{R} V_m$\;
    $\Tilde{x} := r || x || \pad(x), \quad \Tilde{x} \rightarrow x_1 || x_2 || ... || x_t$\;
    $H_0 := IV$\;
    \For{$i \leftarrow 1$ \KwTo $t$}{
        $H_i = f(H_{i-1}, x_i \oplus r)$\;
    }
    
    \Return $(r, g(H_t))$\;
\end{algorithm2e}

\begin{remark}
    Any hash function transforms in randomised, without change in the algorithm.
    All manipulations are made on the input message. 
\end{remark}

\section{HAsh Iterated FrAmework (HAIFA)}

\begin{definition}[HAsh Iterated FrAmework (HAIFA)]
    By Bikam Dankelman (2006).
    \begin{enumerate}
        \item \textbf{compressive function.}:
            $$H_i = f(H_{i-1}, x_i, s, l_i),$$
            where $s$ is randomised salt (if there is no need, then $s=0$),
            $l_i$ is a length of possessed data, (with $x_i$, but without padding).
            $l_i = 0$ is service block (additional data, isn't a block of message
        \item \textbf{Initialisation state.}
            $$D_i = \langle n \rangle || 0 ... 0.$$
            $$H_0 = f(IV, D, 0, 0),$$
            that depends on $n$ (can be precalculated).
        \item \textbf{Padding.} $\pad(x) = 10...0||\langle length(x) \rangle || \langle n \rangle$.
        \item \textbf{Final transformation.} $g(H_t) = \mathrm{trunc}_n(H_t)$, because there is
            a recommendation to use wide-pipe ($w \geqslant 2n$).
    \end{enumerate}
\end{definition}

\section{Other hash functions}

\begin{enumerate}
    \item BLAKE
    \item Skein
    \item Grostle
    \item ``Купина''
    \item Sponge construction
\end{enumerate}

\section{Parametrised hash functions}

Have $\h: \{0, 1\}^{\leqslant l(n)}  \times \mathcal{K} \rightarrow \{0, 1\}^n$,
where $k \in \mathcal{K}$ is an open parameter (salt, vinegar, garlic, etc.).

Cryptographical properties:
\begin{enumerate}
    \item \textbf{Collision resistance (Coll).} ...
    \item \textbf{Preimage resistance (Pre).} $y = \h(x)$ and $\mathcal{K}$ are not
        dependent on analytic.
        \begin{itemize}
            \item[+] \textbf{Always Pre (aPre).} $k$ is chosen randomly by analytic.
            \item[+] \textbf{Everywhere Pre (ePre).} $y$ is chosen by analytic and $k$
                is chosen randomly by analytic.
        \end{itemize}
    \item \textbf{Second preimage resistance (Sec).} $x$ and $k$ are chosen randomly
        by analytic.
        \begin{itemize}
            \item[+] \textbf{Always Sec (aSec).} $k$ is chosen by analytic, $x$ is random.
            \item[+] \textbf{Everywhere Sec (eSec).} $x$ is chosen by analytic, $k$ is random.
        \end{itemize}
\end{enumerate}

\begin{remark}
    For non parametrised hash functions properties ePre, eSec has no meaning.
    For more, aPre and aSec are the same as Pre and Sec correspondent.
\end{remark}

\begin{algorithm2e}[ht]
    \caption{General algorithm for iterative parametrised hash functions}
    % \label{alg:default}
    
    \KwInput{Some message $x$ and key $k$.}
    \KwOutput{hash of message $x$, $\h(x)$, and random salt used in the algorithm.}

    $\Tilde{x} := x || \pad(x) \rightarrow x_1 || x_2 || ... || x_t$\;
    $H_0 := IV(k)$\;
    \For{$i \leftarrow 1$ \KwTo $t$}{
        $H_i = f_k(H_{i-1}, x_i \oplus r)$\;
    }
    
    \Return $g_k(H_t)$\;
\end{algorithm2e}

Parametrised hashing constructions

\begin{enumerate}
    \item \textbf{Classic scheme.} $f \rightarrow f_k$. (Coll, ePre)
        $$H_0 = IV,$$
        $$H_i = f_k(H_{i-1}, x_i), \quad i = \overline{1, t},$$
        $$\h_k(x) = g(H_t).$$
    \item \textbf{Enveloped MD.} (Coll, ePre, PRO(pseudo random oracle))
        $$H_0 = IV_1,$$
        $$H_i = f_k(H_{i-1}, x_i), \quad i = \overline{1, t-1},$$
        $$\h_k(x) = H_t = f_k(IV_2, H_{t-1} || x_t).$$
    \item \textbf{Linear Hash.} Use $t$ different independent keys $k_i$,
        $i = \overline{1, t}$. (ePre, PRO).
        $$H_0 = IV,$$
        $$H_i = f_{k_i}(H_{i-1}, x_i), \quad i = \overline{1, t},$$
        $$\h_k(x) = g(H_t).$$
    \item \textbf{XOR Linear Hash.} Use $t+1$ different independent keys $k_i$,
        $i = \overline{0, t}$. (Coll, ePre, eSec).
        $$H_0 = IV,$$
        $$H_i = f_{k_0}(H_{i-1} \oplus k_i, x_i), \quad i = \overline{1, t-1},$$
        $$\h_k(x) = g(H_t).$$
    \item \textbf{Victor Shoup constructions.} Use $\lceil \log_2(t) \rceil + 1$ different independent
        keys. (Coll, ePre, eSec)
        Let $\mu(i) = \mu_i$ be a max degree of $2$, that divides $i$.
        ($\mu(1) = 0$, $\mu(2) = 1$, $\mu(3) = 0$, $\mu(4) = 2$,
        $\mu(5) = 0$, $\mu(6) = 1$, $\mu(7) = 0$, $\mu(8) = 3$)
        $$H_0 = IV,$$
        $$H_i = f_k(H_{i-1} \oplus k_{\mu(i)}, x_i), \quad i = \overline{1, t},$$
        $$\h_k(x) = g(H_t).$$
    \item \textbf{ROX construction.} Random Oracle XOR.
        (Coll, ePre, aPre, Pre, eSec, aSec, Sec, but PRO).
        Let $\mathcal{K} = V_k$, $l_k$ is a length of the key $k$,
        $l_k \leqslant |x| \leqslant c^{L}$. Use oracles
        $RO_1: V_{l_k} \times V_{l_k} \times V_{\lceil\log_2(L)\rceil} \rightarrow V_w$,
        $RO_2: V_{l_k} \times V_{L} \times V_{\lceil\log_2(L)\rceil} \rightarrow V_{2w}$.
        (in practice, $RO_1$ and $RO_2$ can be build on a one oracule with different input parameters).
        \begin{enumerate}
            \item Padding function:
                $\mathrm{roxpad}(x) = \mathrm{trunc}(RO_2(ae, |x|, 1) || RO_2(ae, |x|, 2) || ...)$,
                where $ae$ are $l_k$ first bits of a message x, $|x|$ is a length encoded in $l_k$ bits.
            \item Hash computation.
                $$\Tilde{x} = x || \mathrm{roxpad}(x) \rightarrow x_1 || x_2 || ... || x_t.$$
                $$r_i = RO_1(k, ae, i), \quad i = \overline{1, \lfloor\log_2(t)\rfloor}.$$
                $$H_0 = IV,$$
                $$H_i = f_k(H_{i-1} \oplus r_{\mu(i)}. x_i),$$
                $$\h_k(x) = H_t.$$
        \end{enumerate}
        
\end{enumerate}

\section{Message authentication code (MAC)}

Message authentication code, mock insertion.
The aim of mac codes is to chech the authenticity of a message
$mac_k: \{0, 1\}^{\leqslant l(n)} \times \mathcal{K} \rightarrow \{0, 1\}^{n}$,
where $k$ is secret.

Cryptographic properties of MAC-codes:
\begin{enumerate}
    \item \textbf{Traditional.} Coll, Pre, ePre, Sec, eSec. (aPre and aSec has no sense,
        because $k$ is a secret key and can't be chosen).
    \item \textbf{Key recovery resistance.} Attack is to find a key $k$ or a part of a key
        by a given set $\{(X_1, mac_k(X_i)) | i = \overline{1, q}\}$.
    \item \textbf{Forgery resistance.} attack is to find $mac_k(x)$ for a message $X \not in \{X_i\}$,
        by a given set $\{(X_1, mac_k(X_i)) | i = \overline{1, q}\}$.
        \begin{enumerate}
            \item \textbf{Existential forgery.} Analytic can't control the value $X$.
            \item \textbf{Selective forgery.} Analytic can choose the value of $X$.
        \end{enumerate}
\end{enumerate}

\begin{definition}[$(\varepsilon, T, q, q', q'', l)$ - stability]
    By Belare, Kilian, Rogevey, $(\varepsilon, T, q, q', q'', l)$ - stability means,
    That the adversary $A$ restricted by time $T$, having messages of length
    $\leqslant l$ and can get:
    \begin{enumerate}[noitemsep]
        \item get $q$ random pairs $(X, mac_k(X))$.
        \item get $q'$ pairs $(X', mac_k(X'))$ with chosen $X'$.
        \item Check correctness of $q''$ pairs $(X'', M'')$. The pair is correct,
            if $M'' = mac_k(X'')$, that $A$ forms for himself.
            ($q''$ can be given by asymptotic $O(\cdot)$. That is proved that
            the change isn't affecting the stability)
        \item $A$ will succeed with probability $\leqslant \varepsilon$.
    \end{enumerate}
\end{definition}

\begin{remark}
    General iterative MAC code structure is the same as the structure of
    general parametrial hash function.
\end{remark}

\begin{remark}
    Absence of non-trivial final transformation is criminal
    (whatever it means, because it must be present).
\end{remark}

\begin{definition}[Outer collision]
    An outer collision is a collision $(x, x')$, $x \neq x'$, $mac_k(x) = mac_k(x')$.
\end{definition}

\begin{definition}[Inner collision]
    An inner collision is a collision $(x, x')$, $x \neq x'$, $H_t = H_t'$.
\end{definition}

\section{Zhu multicollisions}

\begin{definition}[$k$-multicollision]
    $k$-multicollision is a set $\{X_1, X_2, ..., X_k\}$ such that:
    \begin{equation*}
        \h(X_1) = \h(X_2) = ... = \h(X_k).
    \end{equation*}
\end{definition}

Intuitive expected complexity: collision $(X_1, X_2)$ for $O(2^{\frac{n}{2}})$,
and $X_3, ..., X_k$ as preimages for $O(2^n)$, so general complexity is $O(k 2^n)$

The idea of Zhu: Let $J_s : V_w \rightarrow V_m \times V_n$, $J_f(H) = (x, y)$,
$x \neq y$, but $f(H, x) = f(H, y)$.

\begin{theorem}
    $2^t$ multicollision for iterative hash function $\h$ based on MD construction
    with calls of $J_f$.
\end{theorem}

\begin{algorithm2e}[ht]
    \caption{Build $2^t$ multicollision}
    \label{alg:build_multicollision}
    
    % \KwInput{Some message $x$ and key $k$.}
    % \KwOutput{hash of message $x$, $\h(x)$, and random salt used in the algorithm.}

    $H_0 := IV$\;
    \For{$i \leftarrow 0$ \KwTo $t - 1$}{
        $(M_0^{(i)}, M_1^{(i)}) := J_f(H)$\;
        $H = f(H, M_0^{(i)})$\;
    }
    
    \Return $g_k(H_t)$\;
\end{algorithm2e}

\begin{remark}
    Complexity of algorithm \ref{alg:build_multicollision} with use of birthday attack
    is $O(t 2^{\frac{m}{2}}$, where $t$ is complexity of oracle.
\end{remark}

\section{Attacks on a concatenational hash function}

Let $\h_1$, $\h_2$ be two hash functions with outputs $n_1$ and $n_2$, $n_1 \leqslant n_2$.
let $\h(x) = \h_1(x) || \h_2(x)$, (for example $MD5 || SHA1$).

Intuitive expected collision search complexity: $\approx 2^{\frac{n_1 + n_2}{2}}$.

With idea of Zhu:
\begin{enumerate}
    \item Build $2^{\frac{n_1}{2}}$ multicollisions Zou for $\h_1$
        (with use of birthday attack, $O(\frac{n_2}{2} 2^{\frac{n_1}{2}})$)
    \item With a big probability collision for $\h_2$ will be found.
\end{enumerate}
General complexity: $n_2 2^{\frac{n_1}{2} - 1} + 2^{\frac{n_2}{2}}$.

\section{Attack of long messages}

\begin{enumerate}
    \item Attack of Winternitz \ref{alg:attack_of_Winternitz}. (Complexity:$(t+\frac{2^n}{t})$)
        The minimum complexity is at $t = 2^{\frac{n}{2}}$.
        If $t > 2^{\frac{n}{2}}$ complexity rises.

        \begin{algorithm2e}[ht]
            \caption{Build $2^t$ multicollision}
            \label{alg:attack_of_Winternitz}
            
            % \KwInput{Some message $x$ and key $k$.}
            % \KwOutput{hash of message $x$, $\h(x)$, and random salt used in the algorithm.}
        
            Compute all states $\{H_i\}$, $i = \overline{0, t}$\;
            Generate $\frac{2^n}{t}$ random blocks and compute $H^* = f(H_0, y)$\;
            Has a collision $H^* = H$ with probability $p(1) \approx 0.63$ (Yuval)\;
            Has second preimage $y || X_{i+1} || X_{i+2} || ... || X_{t}$
            \For{$i \leftarrow 0$ \KwTo $t - 1$}{
                $(M_0^{(i)}, M_1^{(i)}) := J_f(H)$\;
                $H = f(H, M_0^{(i)})$\;
            }
            
            \Return $g_k(H_t)$\;
        \end{algorithm2e}
    \item Din attack.
    \item Kelsey Schnaer attack.
\end{enumerate}

\begin{theorem}[Prenell van Oorshat]
    For itarative MAC code, inner collision can be found with $q = O(2^{\frac{w}{2}})$
    known texts and with $q'$ chosen texts, where:
    $$q' = \left\{ \begin{array}{cc}
        0               & \text{if $g_k$ is a biection.} \\
        O(2^{w - n})    & \text{else} \\
    \end{array} \right.$$
\end{theorem}

Approaches for Mac codes construction:
\begin{enumerate}
    \item Based on general iterative construction.
    \item Based on hash functions.
    \item Based on special mode of block ciphers or stream ciphers.
    \item Based on a universal hash function.
\end{enumerate}

\section{MAC codes based on hash functions}

\begin{remark}
    Let $\h$ be a hash function, available as oracle.
\end{remark}

\begin{definition}[Tsudik's scheme]
    \begin{enumerate}
        \item Prefix scheme: $mac_k(x) = \h(\hat{k} || x)$, $\hat{k}$ is a key
            padded to full block or blocks.
        \item Suffix scheme: $mac_k(x) = \h(x || k)$.
        \item Enveloped scheme: $mac_{k_1, k_2}(x) = \h(\hat{k_1} || x || k_2)$
            or $mac_{k}(x) = \h(\hat{k} || x || k)$.
    \end{enumerate}
\end{definition}

\begin{remark}
    \begin{enumerate}
        \item If prefix scheme is based on classical MD construction, then
            there might be appliable ``length extension attack''.
            Final transformation make it attack more hard to implement.
            For a Sponge attack, don't work at all (authors recommend that construction).
        \item Suffix scheme is bad (no choice).
        \item There is also attack on the third, enveloped construction.
    \end{enumerate}
\end{remark}

\begin{remark}
    Creators of Grostl recommend the enveloped scheme (They have final transformation $g$).
\end{remark}

\begin{definition}[HMAC by Belare, Capelli, Krawezyk]
    $$HMAC_{k_1, k_2}(x) = \h(\hat{k}_2 || \h(\hat{k}_1 || x)).$$
    $$HMAC_{k}(x) = \h(k \oplus opad || \h(k \oplus ipad || x)).$$
\end{definition}

\begin{remark}
    Extension attack and Prenel-Van-Oorshott are not working.
\end{remark}

\begin{remark}
    HMAC vs EnvelopeMD. EMD use f instead of h in final transform, so EMD is faster. HMAC use h as oracle, so has simpler implementation. 
\end{remark}

\begin{theorem}[$HMAK_{k_1, k_2}$ resilience]
    If $h$ is collision resistant hash function and $H_K(x) := h(k||x)$ is forgery resilient MAC-code with fixed size $|x| = n$, then $HMAC_{k_1, k_2}$is forgery resilient MAC-code for arbitrary $x$ of arbitrary size.
\end{theorem}
\begin{proof}
    Proof by invitational modelling with use of proof from the opposite 
\end{proof}

\begin{remark}
    For $HMAC_{k}(X)$ it is not as simple to prove the resilience, because $k_1$ and $k_2$ are not random in general. There is a need of more strict requirements or belief in resilience.
\end{remark}

\begin{remark}
    RFS 2104, ISO: specification of $HMAC_k$
    \begin{enumerate}
        \item Key alignment $|k| \leqslant m \rightarrow$ extend with sero bits, i.e. $|k| > m \Rightarrow \cap{k} := \h(k)||0...0$, so  $|\cap{k}| = m$.
        \item $opad = 0x5C5C...5C$, $ipad = 0x3636...36$.
    \end{enumerate}
\end{remark}


\begin{definition}[CBC-MAC]
    Iterative MAC-code, write as $CBCMAC_k(x)$.
    \begin{enumerate}
        \item $H_0 := 0^n;$
        \item $H_i := E_k(H_{i-1} \oplus x_i), i = \overline{1, t};$
        \item $T := H_t.$
    \end{enumerate}
\end{definition}

\begin{definition}[CFB-MAC]
    \begin{enumerate}
        Iterative MAC-code, write as $CFBMAC_k(x)$.
        \item $H_0 := 0^n;$
        \item $H_i := E_k(H_{i-1}) \oplus x_i, i = \overline{1, t};$
        \item $T := H_t.$
    \end{enumerate}
\end{definition}

\begin{definition}[RIPE-MAC]
    Iterative MAC-code, write as $RIPEMAC_k(x)$.
    \begin{enumerate}
        \item $H_0 := 0^n;$
        \item $H_i := E_k(H_{i-1} \oplus x_i) \oplus x_i, i = \overline{1, t};$
        \item $T := H_t.$
    \end{enumerate}
\end{definition}

\begin{definition}[SCFB-MAC]
    \begin{enumerate}
        Iterative MAC-code, write as $CFBMAC_k(x)$.
        \item $H_0 := 0^n;$
        \item $H_i := E_k(H_{i-1}) \oplus x_i, i = \overline{1, t};$
        \item $T := H_1 \oplus H_2 \oplus ... \oplus H_t.$
    \end{enumerate}
\end{definition}


