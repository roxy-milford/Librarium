\chapter{Mat stat anal}

\section{Лекція 1. Основні задачі та поняття математичної статистики}

\subsection{Основні задачі математичної статистики.}

\section{Лекція 2. Точкові оцінки та загальні вимоги до них. Властивості точкових оцінок.}

\subsection{Задача оцінювання}

Нехай $\overline{X} = (X_1, ..., X_n)$ --- вибірка з сімї розподілів
$\mathcal{F}_0 = \{ F(c, \theta), \theta \in \Theta \}$, яка залежить від
невідомого параматра $\theta \in \Theta$  --- функція розподілу, функціональний вигляд якої є відомим
з точністю до невідомого параметра


Приклади

1) $\mathcal{F}_0 = \{ N(\theta, 1), \theta \in \mathbb{R} \}$
--- Сімя нормальнрих розподілів з невідомими математичним сподіванням.

1) $\mathcal{F}_0 = \{ N(a, \theta), \theta > 0 \}$ --- сімя нормальних розподілів з невідомою
дисперсією, --- відома константа

3) --- сім'я нормальних розподілів з обома невідомими параметрами


---векторний параметр


Задача оцінювання: використовуючи статистичну інформацію,
що міститься в  , зробити висновки про істинне значення  невідомого параметра моделі

При точковому оцінюванні, шукають таку статистику $T_n = T(X_1, ..., X_n)$,
значення якої при заданій реалізації $(x_1, ..., x_n)$ вибірки $(X_1, ..., X_n)$
приймають за наближине значення параметра $\theta$. При цьому  називають
точковою оцінкою параметра $\theta$.

Які оцінки можна вважати хорошими?

\begin{definition}[змістовною оцінкою параметра]
    Статистика  називаєть ся змістовною оцінкою параметра $\theta$, якщо 
    $$T_n \xrightarrow{P} \theta, \quad n \rightarrow \infty$$

    $$\forall \varepsilon > 0 \quad P(|T_n - \theta|>\varepsilon) \rightarrow 0, \quad n \rightarrow\infty$$
\end{definition}

Сильно змістовна, якщо 
$$T_n \rightarrow \theta, \text{м.н.} , \quad n \rightarrow \infty$$


Статистика  називається незмістовною оцінкою параметра $\theta$, якщо 


Тут запис $M_{\theta}$, $P_{\theta}$, $D_{\theta}$, означає, що відповідні величини
обчислюються в припущенні, що $X_i \sim^d F(x, \theta)$ю

озн
Статистика $T_n$ називається асимптотично незміщеною оцінкою $\theta$,
якщо 


приклади
1)
$F_n(x) = \dfrac{1}{n} \sum\limits_{i=1}^n \mathbb{I}(X_i \leqslant x)$ --- емпірична функція розподілу

$F_n(x)$ --- незміщена і змістовна оцінка істинної функції розподілу. Справді

Змістовність за Законом Великих Чисел

з лінійності математичного сподіванням:


2) 
$\overline{X} = \dfrac{1}{n} (X_1 + ... + X_n)$ --- вибіркове середнє, у припущені, що $MX_i < \infty$,
Тоді $\overline{X}$ змістовна і незміщена оцінка для $\theta = MX_1$.

3) $S^2 = \dfrac{1}{n} \sum\limits_{i=1}^{n} (X_i - \overline{X})^2$,
$S_0^2 = \dfrac{1}{n-1} \sum\limits_{i=1}^{n} (X_i - \overline{X})^2$

$S^2 = \overline{X^2} - (\overline{X})^2$. Звідси за законом великих чисел:

$S_n^2 \xrightarrow[n \rightarrow \infty]{P} MX_1^2 - (MX_1)^2 = DX_1$

$S_n^2$ --- змістовна оцінка для 

--- було перевірено на попередній практиці.


Отже, --- незміщена для 
--- Асимптотично не зміщена оцінка.

твердження
Про змістовність і незміщеність вибіркових моментів

Нехай  --- борелева функція така, що .
Тоді статистика
є незміщеною змістовною оцінкою для параметра .


Зауваження:
Незміщеність є привабливою властивістю, але не обовязковою.
Незміщена оцінка може й не існувати.

Приклад
Не існує незміщеної оцінки
Нехай  --- вибірка об'єму  з розподілу Пуасона з невідомим
параметром $\theta$
Потрібно оцінити

дов
Припустимо, що  --- незміщена оцінка для $g(\theta)$, тобто
$$\forall \theta > 0 \quad M_{\theta} T(X_1) = \dfrac{1}{\theta}$$

Обчислимо $M_{\theta} T(X_1)$:
$$M_{\theta} T(X_1) = \sum\limits_{k=0}^{\infty} T(k) \cdot P_{\theta}(X_1 = k)$$

Тоді:
$$\sum\limits_{k=0}^{\infty} T(k) \cdot P_{\theta}(X_1 = k) = \dfrac{1}{\theta}, \quad \forall \theta > 0$$
$$\sum\limits_{k=0}^{\infty} T(k) \cdot \dfrac{\theta^k}{k!} e^{-\theta} = \dfrac{1}{\theta}, \quad \forall \theta > 0$$
$$\sum\limits_{k=0}^{\infty} T(k) \cdot \dfrac{\theta^{k+1}}{k!} = e^{\theta}, \quad \forall \theta > 0$$

Але рівність 
$$\sum\limits_{k=0}^{\infty} T(k) \cdot \dfrac{\theta^{k+1}}{k!}
= \sum\limits_{m=0}^{\infty} \dfrac{\theta^{m}}{m!} \forall \theta > 0$$

є неможливою, отже не існує такої функції $T(k)$ яка б
задовільняла  і не залежала від $\theta$.

Отже незміщеної оцінки для  не існує.

Зауважимо, що якщо $T_1$ і $T_2$ ---  незміщені
оцінки параметра $\theta$, то
$$T = C_1 T_1 + C_2 T_2, \quad C_1 + C_2 = 1$$

теж є незміщеною оцінкою

Приклад:
Нехай $X_1$, $X_2$, ..., $X_{2n} \sim B(p)$, де $p$ --- невідомий
параметр. Ми знаємо, що $p = MX_1$. Тому в якості оцінки $\hat{p}$
логічно взяти вибіркове середнє.
Нехай
--- змістовні, незміщені оцінки для $p$ю.

Яка тоді оцінка краща?
Ми повинні мати деякий критерій порівняння оцінок.
 
\subsection{Середньоквадратичний критерій порівняння оцінок}

Означення
Величину $M_{\theta}(T(X_1, ..., X_n) - \theta)^2$ називають
середньоквадратичним відхиленням від параметра.

озн
Кажуть, що стптистика $T_1$ є кращою в середньому
квадратичному за оцінку $T_2$, якщо 

1) $\forall \theta \in \Theta: M_{\theta}(T_1 - \theta)^2 \leqslant M_{\theta}(T_2 - \theta)^2$
і
2) $\exists \theta \in \Theta: M_{\theta}(T_1 - \theta)^2 < M_{\theta}(T_2 - \theta)^2$

озн
Величину $b(\theta) = M_{\theta}T(X_1, ..., X_n) - \theta$ називають зміщенням
оцінки $T(X_1, ..., X_n)$.

Зауважимо, що якщо $T = T(X_1, ..., X_n)$ є незміщеною оцінкою , то  і тоді


приклад
Продовження попереднього
Обчислимо і. маємо:
Оскільки, то
а отже є кращою в середньому квадратичному.

\subsection{Оптимальність оцінок}

Позначимо $K_b = K_{b(\theta)}$ --- клас оцінок зі зміщенням $b(\theta)$.
Зокрема $K_0$ --- клас незміщених оцінок.

озн
Оцінка $T^* \in K_b$ називається оптимальною в класі $K_b$,
якщо вона не гірша усіх інших оцінок цього класу $K_b$ в 
середньоквадратичному сенсі, тобто:
$$\forall T \in K_b, \quad \forall \theta \in \Theta \quad
M_{\theta} (T^* - \theta)^2 \leqslant M_{\theta} (T - \theta)^2$$

заув
Оскільки для довільної статистики $T \in K_b$:

і оскільки зміщення в класі  є однаковим,
то оцінка є оптимальною в , якщо 

озн
Оптимальна оцінка в класі  незміщених оцінок
називається оптимальною. (упускаємо слова у класі)

Оптимальна оцінка може не існувати. Але якщо
оптимальна оцінка існує, то вона єдина.

теорема
Про єдиність оптимальної оцінки в
Нехай $T_1 = T_1(X_1, ..., X_n)$ і $T_2 = T_2(X_1, ..., X_n)$ ---
дві оптимальні (в класі ) оцінки. Тоді
дов
Розглянемо оцінку
Оскільки 
то
Оскільки і (за умовою) є оптимальними, то
, де
Обчислимо дисперсію :
За нерівністю Коші-Буняковського:
Отже
Звідси випливає, що  (інакше  і  не були б оптимальними).
Рівність  означає, що в нерівності Коші-Буняковського ((реф))
досягається рівність, а це означає, що  і  є лінійно залежними,
тобто:
Але оскільки  і  , то 
.
Отже: 
м.н.

Приклад
Порівняємо оцінки
і
Для параметра  моделі .
Маємо вибірку  і  є невідомим.
Маємо для :
--- незміщена оцінка.
Для  отримуємо:
Тоді
Отже
Потрібно порівняти  і   .
При  і  середньоквадратичні відхилення співпадають,
а отже жодна не є кращою за іншу. Якщо  , то

Тому при  оцінка  є кращою в середньому квадратичному.

