%---------------------------------------------------------------------
%                   Additional packages for the project
%---------------------------------------------------------------------

%---------------------------------------------------------------------
%                       Page paramethers
%---------------------------------------------------------------------

\usepackage{fullpage}
% \usepackage{indentfirst} % Append indent for the first line in the section
\usepackage{parskip}
\usepackage{url}

%---------------------------------------------------------------------
%                       Graphicx
%---------------------------------------------------------------------

\usepackage{longtable}
\setlength{\LTcapwidth}{\textwidth} % set longtable caption width as \textwidth
\usepackage{array}
\newcolumntype{P}[1]{>{\centering\arraybackslash}p{#1}}
\usepackage{graphicx} % Required for inserting images
\usepackage{multicol}


% \labelformat{table}{\textit{tab.}\,#1}
% \labelformat{figure}{\textit{fig.}\,#1}

%---------------------------------------------------------------------
%                       Math
%---------------------------------------------------------------------
\usepackage{amsmath}
\usepackage{amsfonts}
\usepackage{amssymb}
\usepackage{amsthm}
\usepackage{cancel}
\usepackage{xparse}
\usepackage{faktor}
\usepackage{mathabx}


\makeatletter
\providecommand*{\diff}{\@ifnextchar^{\DIfF}{\DIfF^{}}}
\def\DIfF^#1{\mathop{\mathrm{\mathstrut d}}\nolimits^{#1}\gobblespace}
\def\gobblespace{\futurelet\diffarg\opspace}
\def\opspace{%
    \let\DiffSpace\!%
    \ifx\diffarg(%
        \let\DiffSpace\relax
    \else
    \ifx\diffarg[%
        \let\DiffSpace\relax
    \else
    \ifx\diffarg\{%
        \let\DiffSpace\relax
    \fi\fi\fi\DiffSpace}

%---------------------------------------------------------------------
%                       Math operators
%---------------------------------------------------------------------

\DeclareMathOperator{\Eval}{Eval}
\DeclareMathOperator{\Gen}{Gen}
\DeclareMathOperator{\Enc}{Enc}
\DeclareMathOperator{\Dec}{Dec}
\DeclareMathOperator{\Sig}{Sig}
\DeclareMathOperator{\Verif}{Verif}
\DeclareMathOperator{\Dh}{DH}
\DeclareMathOperator{\KDF}{KDF}
\DeclareMathOperator{\Encode}{Encode}
\DeclareMathOperator{\Hash}{H}

\DeclareMathOperator*{\amper}{\&}
\DeclareMathOperator{\h}{h}
\DeclareMathOperator{\pad}{pad}
\newcommand*{\xor}{\ensuremath \oplus}
\DeclareMathOperator{\rang}{rang}
\DeclareMathOperator{\Ker}{Ker}
\DeclareMathOperator{\im}{Im}
\DeclareRobustCommand{\divby}{\mathrel{\text{\vbox{\baselineskip.65ex\lineskiplimit0pt\hbox{.}\hbox{.}\hbox{.}}}}}


%---------------------------------------------------------------------
%                       Math environments
%---------------------------------------------------------------------
\usepackage{thmtools}                   % for theorem styles and tools
\labelformat{theorem}{(\textit{Thr.}\,#1)}

\declaretheorem[name=Theorem, numberwithin=chapter]{theorem}
\declaretheorem[name=Corollary, numberwithin=theorem]{corollary}
\declaretheorem[name=Claim, numberwithin=chapter]{claim}
\declaretheorem[name=Lemma, numberwithin=chapter]{lemma}
\declaretheorem[name=Definition, numberwithin=section]{definition}
\declaretheorem[name=Remark, style=remark, numberwithin=chapter]{remark}
\declaretheorem[name=Hypothesis, numberwithin=chapter]{hypothesis}
\declaretheorem[name=Axiom, numberwithin=chapter]{axiom}
\declaretheorem[name=Problem, numberwithin=chapter]{problem}
\declaretheorem[name=Attack, numberwithin=chapter]{attack}
\declaretheorem[name=Solution'язання, style=remark, numberwithin=problem]{solution}
\declaretheorem[name=Conclusion, numbered=no]{conclusion}
% \declaretheorem[name=Example, thmbox=M, numberwithin=chapter]{example}
\declaretheorem[name=Example, numberwithin=chapter]{example}

%---------------------------------------------------------------------
%                       Links
%---------------------------------------------------------------------

\usepackage{hyperref}
\hypersetup{
    colorlinks,
    % citecolor = black,
    % filecolor = black,
    % linkcolor = black,
    % urlcolor = blue
    % anchorcolor = black,
    % menucolor = red,
    % runcolor = cyan,
    allcolors = black
}
\usepackage{minitoc} % small table of contsnts for every chapter
\usepackage{shorttoc}

% \labelformat{algocf}{(\textit{alg.}\,#1)} % reference label for algorithms
% \labelformat{table}{(\textit{tab.}\,#1)}
% \labelformat{figure}{(\textit{fig.}\,#1)}

%---------------------------------------------------------------------
%                       Enumeraion
%---------------------------------------------------------------------

\usepackage[inline]{enumitem}
\setlist[itemize]{noitemsep, topsep=0pt}
\setlist[enumerate]{noitemsep, topsep=0pt}
%---------------------------------------------------------------------
%                       Images
%---------------------------------------------------------------------
\usepackage{tikz}                       % for beautiful math images
\usetikzlibrary{patterns}				% for field figurs
\usepackage{pgfplots}					% for function ploting
\pgfplotsset{compat = newest}			% use last version of {pgfplots}
\usepgfplotslibrary{polar}				% for polar system
\usepgfplotslibrary{fillbetween}
\usepackage{wrapfig} 					% for wrapping the text
\usepackage{graphicx}
\usepackage{calc}


%-------------------------------------------------------------------------------
%                       Chapter head format
%-------------------------------------------------------------------------------
% \makeatletter
% \def\@makechapterhead#1{%
%   \vspace*{-40\p@}%
%   {\parindent \z@ \filleft \normalfont
%     \ifnum \c@secnumdepth >\m@ne
%       \if@mainmatter
%         %\huge\bfseries \@chapapp\space \thechapter
%        \LARGE\bfseries {\color{gray}\thechapter }\space%
%         %\par\nobreak
%         %\vskip 20\p@
%       \fi
%     \fi
%     \interlinepenalty\@M
%     \LARGE\color{reddy} \bfseries #1\par\nobreak
%     \vskip 20\p@
%   }}
% \makeatother


% \makeatletter
% \def\@makechapterhead#1{%
%     {\parindent \z@ \raggedleft \normalfont
%         \ifnum \c@secnumdepth >\m@ne
%             \huge\bfseries \thechapter.\ % <-- Chapter # (without "Chapter")
%         \fi
%         \interlinepenalty\@M
%         #1\par\nobreak% <------------------ Chapter title
%         \vskip 40\p@% <------------------ Space between chapter title and first paragraph
%     }}
% \makeatother
%---------------------------------------------------------------------
%                       End
%---------------------------------------------------------------------